\documentclass{article}
\usepackage[spanish, es-tabla]{babel}

\usepackage[utf8]{inputenc}

%\usepackage[utf8x]{inputenc}
\usepackage[T1]{fontenc}


%% Sets page size and margins
\usepackage[letterpaper,top=2.5cm,bottom=2cm,left=3cm,right=3cm,marginparwidth=1.75cm,  headsep=0.2cm, headheight=2cm]{geometry}

%% Useful packages
\usepackage{verbatim} % citar
\usepackage{import}   %importar
\usepackage{amsmath}
\usepackage{float}
\usepackage{graphicx}
\usepackage{multirow}
\usepackage{booktabs, makecell}
\usepackage[table,xcdraw]{xcolor}
\usepackage[colorinlistoftodos]{todonotes}
\usepackage[colorlinks=true, allcolors=blue]{hyperref}
\usepackage[hang, small,up,textfont=it,up]{caption} 
\usepackage{fancyhdr}
\usepackage{tabularx}
\renewcommand{\tablename}{Tabla}
\usepackage{authblk}
\usepackage{csquotes}    %citas
\usepackage{tikz}         %arboles
\usetikzlibrary{trees}

\renewenvironment{quote}
  {\small\list{}{\rightmargin=3.5cm \leftmargin=3.5cm}%
   \item\relax}
  {\endlist}

% Marca de Agua
%\usepackage{draftwatermark}
%\SetWatermarkText{Borrador}
%\SetWatermarkScale{4}

% Enumeración
\usepackage{enumitem}
%\setenumerate[1]{label=\thesubsection.\arabic*}

%\setenumerate[2]{label*=\arabic*.}
%\makeatletter
%\renewcommand{\@seccntformat}[1]{}
%\makeatother

%cabecera y pies
\pagestyle{fancy}
\fancyhf{}
\rhead{\includegraphics[width=0.12\textwidth]{SAMU_e.png}}
\lhead{
Análisis Red Unidades Emergencias SSVQ\\ 
Dr E. Céspedes, Tecnologías SAMU \\
SAMU V Región \\
2019}
\rfoot{P\'agina \thepage}


% titulo documento
\title{Análisis Red Unidades Emergencias SSVQ}
\author{Dr E Céspedes}
\date{Abril 2019}


\begin{document}
\maketitle


\tableofcontents

\section{Introducción}
El Servicio de Salud Viña del Mar-Quillota es uno de los tres Servicios de Salud(SS) ubicados en la Región de Valparaíso. La cantidad de gente y superficie terriorial de la que es responsable lo ubica en los primeros lugares dentro de todos los SS del país.

Dentro de las distintas métricas existentes para los Servicios de Urgencia, aún no se cuenta con indicadores fiables tanto por la variabilidad de en sus cálculos, dado la dependecia de personas en ello, como lo poco práctico de sus lecturas en los existentes.

El siguiente informe es una descripicón general realizada de manera tranversal a todos los establecimientos hospitalarios que cuentan con unidades de emergencias. La mayor cantidad de atenciones de urgencias suceden en hospitales, pero parte de la población que se atiende en sistema privado o municipal no fue incluído en este informe dado que no se poseen sábanas de datos que permitan analizar información.

El objetivo del informe es doble: 
\begin{itemize}
\item Mostrar la posibilidad al tener registros electrónicos de acceder y analizar la información de manera eficaz, confiable, tranversal y célere.
\item Exponer por medio de un documento formal la realidad tranversal de las unidades de emergencias del SS, dando la posbilidad de realizar comparaciones dado que se utilizaron las mismas herramientas para los cálculos y las descripciones.
\end{itemize}



\section{Metodología}
Estudio retrospectivo de corte transversal. El universo fueron todos los registros de urgencia encontrados en sistema Track Care (SIDRA) con el que se realiza atención de urgencia en el Servicio Salud Viña del Mar Quillota durante el año 2018.

Los criterios de inclusión fue existir en los registros de SIDRA. Se excluyeron los errores administrativos de inscripción.

Se analizó de manera crítica los datos para que entradas inútiles o variables confundentes que ensuciaran los resultados. Se crearon variables especiales que recopilaban situaciones de interes clínico en base a otras existentes.

Se utilizó medidas de tendencia central y de dispersión. Se evitó la utilización de máximos y mínimos en la descripción del los datos dado la posibilidad de que los valores extremos correspondan a errores durante la recolección de datos en la atenci' de emergencia relacionados con la informática.

Se descargó desde la página del aplicativo los reportes. El análisis de las bases de datos, corrección de estas y el análisis estadístico se realizó íntegramente bajo herrmientas GNU. Se utilizó Linux, lenguaje Python versión 3.2 sobre el IDE Jupyter Notebook versión 1.40.16. La presentación de resultados se realizó con lenguaje \LaTeX versión 1.40.16. 


\section{Resultados}



 
\subsection{Análisis del SSVQ}


\input{../resultados/compilado_final/01SS.final}



\subsection{Análisis por hospital}

\input{../resultados/compilado_final/Hospitales1.final}








 
\end{document}

